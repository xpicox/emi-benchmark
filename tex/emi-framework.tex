\documentclass[convert]{standalone}
\usepackage{multirow}
\usepackage{tikz}
\usetikzlibrary{shapes.geometric,arrows.meta,chains}

\newlength{\starheight}
\settoheight{\starheight}{A}
\newcommand*{\letterrank}[2]{%
  \begin{tikzpicture}[baseline]
  \node[inner sep=0pt,outer sep=0pt,minimum width=\starheight,%text width=1.1em,% minimum width=1em,
  anchor=base,font=\sffamily,align=center] at (\starheight*0.25,0) {#1};
  \foreach \i in {1,...,3} {
    \pgfmathparse{\i<=#2 ? "fill" : "draw"}
    \edef\content{\pgfmathresult}
    \node[star,draw,\content,star point ratio=2.3,anchor=outer point 3,inner sep=0pt,minimum width=\starheight] at (\i*\starheight*1.25,0) {};
  }
  \end{tikzpicture}%
}
\newcommand{\E}[1]{\letterrank{E}{#1}}% efficiency
\newcommand{\M}[1]{\letterrank{M}{#1}}% modularity
\newcommand{\I}[1]{\letterrank{I}{#1}}% isolation
\def\arraystretch{1.4}%
\def\rsep{-2ex}%
\begin{document}
\begin{tabular}{c|c|c|c|c|}
 % \cline{3-5}
  \multicolumn{2}{c|}{}&\multicolumn{3}{c|}{\textbf{Distribution}}\\
  \cline{3-5}
  \multicolumn{2}{c|}{}&Internal&Adjacent&External\\
  \hline
  \multirow{6}{*}{\rotatebox[origin=c]{90}{~~~~~~~~\textbf{Generality}}}
  &          &\E{2}&\E{2}&\E{1}\\[\rsep]
  &Parametric&\M{2}&\M{3}&\M{3}\\[\rsep]
  &          &\I{1}&\I{2}&\I{3}\\
  \cline{2-5}
  &          &\E{3}&\E{2}&\E{1}\\[\rsep]
  &Ad-hoc    &\M{1}&\M{2}&\M{2}\\[\rsep]
  &          &\I{1}&\I{2}&\I{3}\\
  \hline
\end{tabular}
\end{document}

%%% Local Variables:
%%% TeX-command-extra-options: "-shell-escape"
%%% End:
